\documentclass[letterpaper,11pt]{article}

\usepackage[svgnames]{xcolor}
\usepackage[utf8]{inputenc}
\usepackage{hyperref}
\hypersetup{
    colorlinks,
    linkcolor={red!50!black},
    citecolor={blue!50!black},
    urlcolor={blue!80!black}
}

\pagestyle{empty}

\definecolor{shadecolor}{gray}{0.93}  % Inner background color of title bars

\raggedright
\setlength{\evensidemargin}{-0.25in}
\setlength{\headheight}{0in}
\setlength{\headsep}{0in}
\setlength{\oddsidemargin}{-0.25in}
\setlength{\paperheight}{11in}
\setlength{\paperwidth}{8.5in}
\setlength{\tabcolsep}{0in}
\setlength{\textheight}{9.5in}
\setlength{\textwidth}{7in}
\setlength{\topmargin}{-0.3in}
\setlength{\topskip}{0in}
\setlength{\voffset}{0.1in}

\newcommand{\resheading}[1]{\vspace{12pt}\setlength{\fboxsep}{0pt}\framebox[\textwidth][l]{\setlength{\fboxsep}{4pt}\fcolorbox{shadecolor}{shadecolor}{\textbf{\sffamily{\mbox{~}\makebox[6.762in][l]{\large #1} \vphantom{p\^{E}}}}}}}

\newcommand{\ressubheading}[4]{
\begin{tabular*}{6.5in}{l@{\extracolsep{\fill}}r}
    \textbf{#1} & #2 \\
    \textit{#3} & \textit{#4}
\end{tabular*}\vspace{-6pt}}

\begin{document}

\begin{tabular*}{7in}{l@{\extracolsep{\fill}}r}
\textbf{\Large Gustavo Zambonin} & \texttt{+}55 (48) 9696 3133 \\
Av. Francisco Roberto da Silva, 925 &
\texttt{gustavo.zambonin at grad.ufsc.br} \\
Biguaçu - SC (88160-284) &
\texttt{\href{http://github.com/zambonin}{github.com/zambonin}}
\end{tabular*}\vspace{8pt}

\resheading{Formação acadêmica}
\begin{itemize}
    \item \ressubheading{Universidade Federal de Santa Catarina}
    {Florianópolis, SC}
    {Bacharelado em Ciência da Computação}
    {2013 - atualmente}
    \begin{itemize}
        \item Previsão para conclusão: 2018
    \end{itemize}
\end{itemize}

\resheading{Experiência profissional}
\begin{itemize}
    \item \ressubheading{Monitoria}
    {UFSC}
    {Probabilidade e Estatística}
    {2014 - 2015}
    \begin{itemize}
        \item Análise exploratória de dados, cálculo de probabilidades de
        eventos, variáveis aleatórias discretas e contínuas, distribuições
        amostrais e estimação de parâmetros.
    \end{itemize}
    \item \ressubheading{Monitoria}
    {UFSC}
    {Introdução à Ciência da Computação}
    {2015}
    \begin{itemize}
        \item Implementação de algoritmos em linguagens de programação (C,
        Java, Pascal, Python), estruturas de seleção e repetição, declaração e
        indexação de variáveis, compilação e execução de programas, entrada e
        saída de dados.
        \item Ementa equivalente ou similar a Computação Científica I,
        Introdução à Computação, Programação Orientada a Objetos I e Introdução
        à Programação Orientada a Objetos.
    \end{itemize}
\end{itemize}

\resheading{Qualificações e atividades complementares}
\begin{itemize}
    \item Inglês avançado (ótima comunicação escrita e boa comunicação oral).
    \item Ministrante do minicurso \textit{A análise de dados usando o
    SEstatNet (Sistema de Ensino-Aprendizagem de Estatística via web)} na 13ª
    SEPEX – Semana de Ensino, Pesquisa e Extensão da UFSC (4h).
    \item Conhecimento intermediário em: \texttt{bash}, C, C\texttt{++}, Java,
    \LaTeX, Python.
    \begin{itemize}
        \item \vspace{-6pt} \textit{Frameworks} para Python:
        \href{https://github.com/burnash/gspread}{gspread},
        \href{http://matplotlib.org/}{matplotlib},
        \href{http://www.numpy.org/}{NumPy},
        \href{http://docs.python-requests.org/en/latest/#}{Requests},
        \href{http://www.scipy.org/}{SciPy},
        \href{http://scrapy.org/}{Scrapy}.
    \end{itemize}
    \item Conhecimento básico em: \\
    Assembly MIPS, AWK, \texttt{bc}, Haskell, HTML, JavaScript, Julia, Pascal,
    Prolog, \texttt{sed}, VHDL.
\end{itemize}

\end{document}
