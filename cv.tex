\documentclass[12pt]{article}

\usepackage{bibentry}
\usepackage[english]{babel}
\usepackage[a4paper, top=1cm, bottom=1cm, left=2.5cm, right=2.5cm]{geometry}
\usepackage[utf8]{inputenc}
\usepackage[OT1]{fontenc}
\usepackage[datesep={}]{datetime2}
\makeatletter\let\saved@bibitem\@bibitem\makeatother
\usepackage[colorlinks, urlcolor={purple!75!black}, plainpages=true]{hyperref}
\makeatletter\let\@bibitem\saved@bibitem\makeatother
\usepackage[sfdefault]{quattrocento}
\usepackage{array, multirow, parskip, xcolor, pifont}

\nobibliography*

\begin{filecontents*}{\jobname.bib}
@misc{Zambonin:202412,
 author = {Gustavo Zambonin and Ricardo Custódio and Lucia Moura
    and Daniel Panario},
 title = {{Faster combinatorial primitives for efficient hash-based
    signatures}},
 note = {In preparation},
}

@misc{Barbosa:202412,
 author = {João Pedro Cardoso Barbosa and Gustavo Zambonin
    and Thaís Bardini Idalino and Ricardo Custódio},
 title = {{Practical Implementation of a Post-Quantum E-voting Protocol}},
 note = {In preparation},
}

@misc{Kamers:202412,
 author = {Anthony Bernardo Kamers and Paola de Oliveira Abel
    and Thaís Bardini Idalino and Gustavo Zambonin and Jean Everson Martina},
 title = {{Practical algorithms and parameters for modification-tolerant
    signature scheme (extended version)}},
 note = {Submitted to the Journal of the Brazilian Computer Society},
}

@misc{Martina:202412,
 author = {Jean Everson Martina and Larissa Gremelmaier Rosa
    and Gustavo Zambonin},
 title = {{From Cat Videos to Catfish: The Case for a New Social Authentication
    Era}},
 note = {Accepted to SPW 2025 (Twenty-ninth International Workshop on Security
    Protocols)},
}

@misc{Silvano:202412,
 author = {Wellington Silvano and Lucas Mayr and Gustavo Zambonin
    and Ricardo Custódio},
 title = {{Balancing Transparency, Immutability, and Secrecy in Blockchain:
    Extending Shannon’s Secrecy}},
 note = {In preparation},
}

@misc{Rosa:202412,
 author = {Larissa Gremelmaier Rosa and Gustavo Zambonin
    and Jean Everson Martina},
 title = {{Enhanced SIM swap security practices via ceremony modeling}},
 note = {Submitted to AINA 2025 (39th International Conference on Advanced
    Information Networking and Applications)},
}

@misc{Mayr:202408,
 doi = {10.48550/arXiv.2208.03951},
 author = {Lucas Mayr and Gustavo Zambonin and Frederico Schardong
    and Ricardo Custódio},
 title = {{One-Time Certificates for Reliable and Secure Document Signing}},
 year = 2024,
 month = aug,
 note = {\url{https://doi.org/10.48550/arXiv.2208.03951}},
}

@inproceedings{Kamers:202409,
 doi = {10.5753/sbseg.2024.241677},
 author = {Anthony Bernardo Kamers and Paola de Oliveira Abel
    and Thaís Bardini Idalino and Gustavo Zambonin and Jean Everson Martina},
 title = {{Practical algorithms and parameters for modification-tolerant
    signature scheme}},
 year = 2024,
 month = sep,
 booktitle = {{Proceedings of the 24th Brazilian Symposium on Information and
    Computational Systems Security (SBSeg 2024)}},
 editor = {Altair Santin and Raphael Machado},
 pages = {522--537},
 note = {\url{https://doi.org/10.5753/sbseg.2024.241677}},
}

@article{Biage:202401,
 doi = {10.1109/ACCESS.2024.3358670},
 author = {Gustavo de Castro Biage and Gustavo Zambonin
    and Thaís Bardini Idalino and Daniel Panario and Ricardo Custódio},
 title = {{A concrete LIP-based KEM with simple lattices}},
 year = 2024,
 month = jan,
 journal = {{IEEE Access}},
 volume = 12,
 pages = {16408--16420},
 note = {\url{https://doi.org/10.1109/ACCESS.2024.3358670}},
}

@article{Mayr:202309,
 doi = {10.3390/info14100523},
 author = {Lucas Mayr and Lucas Palma and Gustavo Zambonin
    and Wellington Silvano and Ricardo Custódio},
 title = {{Monitoring key pair usage through distributed ledgers and one-time
    signatures}},
 year = 2023,
 month = sep,
 journal = {{Information}},
 volume = 14,
 number = 10,
 pages = {523--537},
 note = {\url{https://doi.org/10.3390/info14100523}},
}

@article{Perin:202106,
 doi = {10.1007/s13389-021-00264-9},
 author = {Lucas Pandolfo Perin and Gustavo Zambonin and Ricardo Custódio
    and Lucia Moura and Daniel Panario},
 title = {{Improved constant-sum encodings for hash-based signatures}},
 year = 2021,
 month = jun,
 journal = {{Journal of Cryptographic Engineering}},
 volume = 11,
 number = 4,
 pages = {329--351},
 note = {\url{https://doi.org/10.1007/s13389-021-00264-9}},
}

@inproceedings{Zambonin:201907,
 doi = {10.1007/978-3-030-23696-0_20},
 author = {Gustavo Zambonin and Matheus Silva Pinheiro Bittencourt
     and Ricardo Custódio},
 title = {{Handling Vinegar Variables to Shorten Rainbow Private Keys}},
 year = 2019,
 month = jul,
 booktitle = {{Progress in Cryptology -- AFRICACRYPT 2019}},
 editor = {Johannes Buchmann and Abderrahmane Nitaj and Tajjeeddine Rachidi},
 pages = {391--408},
 series = {Lecture Notes in Computer Science},
 volume = 11627,
 note = {\url{https://doi.org/10.1007/978-3-030-23696-0_20}},
}

@inproceedings{Perin:201806,
 doi = {10.1109/ISCC.2018.8538642},
 author = {Lucas Pandolfo Perin and Gustavo Zambonin
    and Douglas Marcelino Beppler Martins and Ricardo Custódio
    and Jean Everson Martina},
 title = {{Tuning the Winternitz Hash-Based Digital Signature Scheme}},
 year = 2018,
 month = jun,
 booktitle = {{2018 IEEE Symposium on Computers and Communications (ISCC)}},
 pages = {537--542},
 note = {\url{https://doi.org/10.1109/ISCC.2018.8538642}},
}
\end{filecontents*}

\newcommand*{\ruleline}[2]{%
    \makebox[\linewidth]{#1{\color{purple!75!black}
        \hspace{2pt}
        \hrulefill
        \hspace{2pt}
        \normalsize{\ding{111}}}}}
\newcommand*{\sep}{$\cdot\;$}

\newcommand*{\rulesection}[1]{\subsection*{\ruleline{#1}}\vspace{-1pt}}

\newenvironment{headertable}
  {\renewcommand{\arraystretch}{1.4}
   \newcolumntype{L}{>{\bf \raggedright}p{0.14\textwidth}}
   \newcolumntype{R}{p{0.82\textwidth}}
   \begin{tabular}{@{\hspace{0mm}}LR}}
  {\end{tabular}}

\newcommand*{\headerrow}[2]{#1 & #2 \\}

\newenvironment{contenttable}[1]
  {\rulesection{#1}
   \renewcommand{\arraystretch}{1.4}
   \newcolumntype{L}{p{0.82\textwidth}}
   \newcolumntype{R}{>{\raggedleft\arraybackslash}p{0.14\textwidth}}
   \begin{tabular}{@{\hspace{0mm}}LR}}
  {\end{tabular}}

\newcommand{\contentrow}[5]{
  \emph{#1} #2 & \multirow[t]{2}{58pt}{#3--#4} \\
  \small{#5} & \\
}

\pagenumbering{gobble}

\begin{document}

\section*{\ruleline{Gustavo Zambonin}}

\begin{headertable}
  \headerrow{About \\[1ex] \scriptsize{(rev. \DTMtoday)}}{
    I am an information security consultant with 7+ years of experience and
    a solid academic background. I was a former lead of technical research and
    development for the Brazilian Digital Signature Standard. I specialize in
    quantum-safe cryptography and public-key infrastructures.
  }

  \headerrow{Address}{
    \href{https://zambonin.org}{\texttt{zambonin.org}}
    \sep \href{mailto:zambonin@pm.me}{\texttt{zambonin@pm.me}}
  }

  \headerrow{Languages}{
    Portuguese (native), English (fluent), French (beginner)
  }
\end{headertable}

\begin{contenttable}{Education}
  \contentrow
    {PhD in Computer Science}
    {
      at UFSC
    }
    {Mar/2024}
    {Today}
    {
      Currently researching novel combinatorial (un)ranking algorithms to
      generate random objects in quantum-safe cryptosystems.
    }
  \contentrow
    {MSc in Computer Science}
    {
      from UFSC (thesis:
      ``\href{https://repositorio.ufsc.br/handle/123456789/219497}{On the
      randomness of Rainbow signatures}'')
    }
    {Aug/2018}
    {Sep/2020}
    {
      I was a visiting researcher at Carleton University under
      a \href{http://archive.is/RHxm4}{Mitacs-CALAREO Globalink Research
      Award}, and a teaching assistant at UFSC that taught order theory,
      lattice theory and algebraic structures.
    }

  \contentrow
    {BSc in Computer Science}
    {
      from UFSC (thesis:
      ``\href{https://repositorio.ufsc.br/handle/123456789/187875}{Performance
      optimization for the Winternitz signature scheme}'')
    }
    {Mar/2013}
    {Jul/2018}
    {
      I was a teaching assistant for a probability and statistics class as
      a sophomore. Later, as a junior, I started working at the Computer
      Security Laboratory, developing features for the Brazilian Digital
      Signature Standard official implementation.
    }
\end{contenttable}

\rulesection{Publications}

\bibentry{Zambonin:202412}.

\bibentry{Barbosa:202412}.

\bibentry{Silvano:202412}.

\bibentry{Kamers:202412}.

\bibentry{Rosa:202412}.

\bibentry{Martina:202412}.

\bibentry{Mayr:202408}.

\bibentry{Kamers:202409}.

\bibentry{Biage:202401}.

\bibentry{Mayr:202309}.

\bibentry{Perin:202106}.

\bibentry{Zambonin:201907}.

\bibentry{Perin:201806}.

\begin{contenttable}{Professional experience}
  \contentrow
    {Information security specialist}
    {
      in partnership with several institutions
    }
    {Sep/2017}
    {Today}
    {
      Some of my roles include acting as a consultant on digital signature
      standards; a ceremony operator deploying e-voting platforms;
      a quantum-safe blockchain researcher; and a computer forensic examiner
      measuring the accuracy of pictures from speed enforcement cameras.
      Maybe I can also help you, get in touch!
    }

  \contentrow
    {Technical lead and researcher}
    {
      at the Computer Security Lab of the Universidade Federal de Santa
      Catarina (UFSC)
    }
    {May/2016}
    {Feb/2024}
    {
      From 2020 onwards, I led the team whose job is to improve, maintain and
      add features to the Brazilian Digital Signature Standard official
      implementation, all derived applications, and normative documents. As
      a result, any Brazilian citizen is able to generate and verify digitally
      signed files per the latest standards.
      \vspace{1ex}

      Until 2019, as a software developer at that same team, I implemented
      several large new features to the signature verification service, such as
      a new responsive web interface, a REST API, and support for verification
      of CMS signatures.
    }
\end{contenttable}

\rulesection{Personal values and interests}

I strive to solve problems and deliver elegant solutions with great efficiency,
attention to detail, and a minimal number of tools---most likely AWK, Bash,
tmux and Vim.

I'm also committed to bring out the best of the people working alongside me,
through frequent knowledge transfers and a constant feedback loop.

I'm enthusiastic about astronomy, immersive sim games, IBM keyboards
specifically older than myself and most songs with a saxophone line. 8)

\bibliographystyle{abbrv}
\nobibliography{\jobname}

\end{document}
